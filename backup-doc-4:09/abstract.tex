\chapter*{Resumo}
%\addcontentsline{toc}{chapter}{Resumo}

Dispositivos robóticos programáveis como \textit{Autonomous Underwater Vehicles} (AUVs) são excelentes meios para exploração subaquática, já que são capazes de executar missões de longa duração com variadas possibilidades de aplicação e objetivos. Neste sentido, o conceito de mula AUV surgiu como mecanismo útil que periodicamente recolhe dados dos AUVs em missão. Para que tal seja possível, é necessário implementar um sistema de localização e posicionamento robusto que permite aos AUVs encontrarem outros veículos de forma a aproximarem-se deles eficientemente.

A presente dissertação foca-se na implementação de um sistema que estima a posição relativa entre AUVs através do método \textit{Ultra-Short Baseline} (USBL). Esta técnica baseia-se na determinação da diferença de fases entre sinais recebidos por um vetor de hidrofones.

Após a implementação e validação do sistema referido, este foi integrado num mecanismo existente que adquire e processa dados de quatro hidrofones. Na fase final, serão executados testes de campo e experiências num ambiente enclausurado, como o tanque do DEEC, seguido de um teste em ambiente real, em mar aberto.


\chapter*{Abstract}
%\addcontentsline{toc}{chapter}{Abstract}

Robotic programmable devices such as Autonomous Underwater Vehicles (AUVs) are great means for underwater exploration, as they are capable of executing long term missions with many possible applications and goals. In this regard, the concept of mule AUVs arises as a valuable mechanism to periodically collect data from survey AUVs during the missions. In order to achieve this, a robust localization and positioning system needs to be implemented allowing the mule AUV to find the other vehicle and draw near it efficiently.

The present dissertation focuses on the implementation of a system which estimates the relative position between AUVs through the Ultra-Short Baseline (USBL) method. The technique relies on accurate estimation of the phase difference between signals received in a hydrophone array. 

After implementation and validation of the mentioned system, it will be integrated in an existing mechanism which was specifically designed to acquire and process data from four hydrophones. In the final stage, field tests and experiments will be executed in a closed environment such as DEEC's tank, followed by a real environment test in open sea.