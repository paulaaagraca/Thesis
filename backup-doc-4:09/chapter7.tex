\chapter{Conclusions}  \label{chap:conclusion}

\section{Summary}

\section{Contributions }

The main contributions of this dissertation are: 
\begin{itemize}
	\item A HDL system design with strict area constraints that receives signals from four different sensors and calculates the phase differences between them
	\item An estimation algorithm for the angle of arrival of an acoustic signal to a configuration of four sensors
	\item An algorithm that chooses dynamically the best configuration of hydrophones, among a set of possible combinations, for a received signal with a certain angle of arrival
\end{itemize}

%Additionally, a scientific article will be written and submitted to the OCEANS Conference that is occurring in May of 2021. 

\section{Future Work}

The developed work arose various innovative ideas and topics according to the state of the art on localization optimization methods. Therefore, there is the intent to produce a a scientific article on the subject to be submitted in the OCEANS Conference 2021.

The adopted optimization techniques in the present work are the starting point of a topic which is not very explored in the current literature. Consequently, there are several areas that can be improved as follows:

\begin{itemize}
	\item Implementing the developed algorithm for AoA estimation as a real-time system which can be used for decision making during underwater navigation
	\item Using Machine Learning techniques in order to find the optimal hydrophone array configuration, to be positioned in an underwater vehicle
	\item Integrating the estimator with a Kalman Filter or a Particle Swarm Optimization approach
\end{itemize}