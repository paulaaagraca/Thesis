\chapter{Proposed System} \label{chap:proposed_sys}

This chapter is dedicated to the presentation and overall explanation of the developed system, highlighting its capabilities, the used methodologies and overall design strategies. 
The system will be presented in two distinct sections. The first component is the HDL module, which falls into the spectrum of hardware design and requires insight on hardware development and good practices. The second section relies on software development to complement the functionality of the mentioned module, so that is possible to deliver the desired result.

\section{HDL Module Design}

The system which is proposed to be implemented in this research work has as input 4 signals which are received by  hydrophone of the array, and outputs an average phase difference between all combinations of pairs of hydrophones. 

- ver regras basicas de hardware development
- sustema sincrono, available clock cycles globais
- hardware limitations
- tamanho das entradas


\begin{enumerate}
	\item Hilbert Filter
	\item Cordic
	\item phasediff
	\item phasemean
\end{enumerate}

\subsection{Hilbert Filter}
- matematica brevemente, equação base, resposta impulsional, ganho, coeficientes e ordem usada
\\
- schematics 
\\
- explicar design decisions
\\
- descrever brevemente flow do sinal no hardware

\subsection{Cordic}
- descriçao do que faz, matematica (?)
\\
- entradas e saídas, clocks, ROM

\subsection{phasediff}
-pequeno esquema 
\\
- 1 sub

\subsection{phasemean}
- pequeno esquema
\\
- N accumulated
\\
---------
\\
apresentar esquema global menos pormenorizado

\section{Angle of Arrival Calculation}

- explicar ambiguidade de fase do sinal

- explicar calculos de angulo de chegada (scipt + matematica) com base nas posições dos hydrophones e do angulo de chegada

- por simulação, conclui que para distancias muito longe (quantizar) não faz diferença ter o TOA e basta os TDOA

