\chapter{Problem Characterization} \label{chap:prob_statement}

This chapter intends to clarify the problem addressed by the present dissertation. The contents explained in the \nameref{chap:sota} chapter were used as initial knowledge basis.

\section{Problem Statement}

As previously mentioned, it is considered a scenario where an AUV is taking part on a long-term underwater mission. In practical terms, the moving survey AUV sends known signals periodically to the surface with a pinger and the mule AUV, with an array of hydrophones, receives the signal and estimates the position of the other AUV to navigate near it.

%___________________________________________

This partial system was developed in previous dissertations and research work, which can be better understood in \cite{afonso-thesis}. Briefly, the system consists on a transducer of four hydrophones forming a 3D array deployed on the mule AUV. All components of this array will receive the same signal wave front. The system then calculates the cross-correlation between the received and expected signals, which is a BPSK modulated binary sequence. The cross-correlation peak indicates the distance between AUVs and it is calculated with timing resolution corresponding to 1 sampling period of the acquired signal, which in the developed systems corresponds approximately to 6mm (with a sampling frequency of 244kHz).


Due to the limitations in dimension of the AUV that will integrate this system, the hydrophones will be placed within short distance from each other, in the range of a few centimeters between them. Due to this restriction, the time resolution obtained by using only the cross-correlation, corresponding to a maximum distance accuracy of approximately 6mm, will not be enough for a reliable estimation of the angle of arrival of the sound wave. Thus, the objective of this work is to refine this measurement by additionally calculating the phase differences of the arriving signals to each hydrophone. 
