\chapter{Introduction} \label{chap:intro}

This chapter intends to specify the context of the present dissertation, describing the considered scenario, technologies and conditions in which the proposed solution is useful. Based on this, the two main goals of the present research work are established. Lastly, it is explained how the document is structured, including a concise summary of each chapter's content.

%-------------------------------------------------------------------------------------------
\section{Context and Motivation} \label{sec:context}

Today, the deep blue ocean still represents a relevant topic of research in the scientific community as it constantly rises new unexplained mysteries. Up to now, only 15\% of the entire ocean floor is mapped based on collected data \cite{deeperblue}. As such, it seems essential to create efficient research tools to improve the discovery of information.

Robotic autonomous underwater vehicles (AUVs) are great means for diverse applications in underwater exploration using variable resource requirements and duration, such as monitoring structures installed in shallow waters or exploring the deep ocean floor for scientific purposes. Particularly in long-term missions, the AUV usually navigates underwater, resorting to docking systems to allow extended navigation periods, until the end of the mission when it returns to the base station. Thus far, the data that is being collected is typically not accessible by any processing system. 

A method that is used to resolve this limitation is employing additional mule AUVs, whose goal is to travel near the survey AUV, collect its data during the mission's term and return in a relatively short time period. This allows the data to be periodically processed during the mission, which facilitates the definition of future courses for the mission, such as shortening its duration or sending additional commands. In the mentioned localization system, high precision is key as it allows the AUVs to reach very short distances between them when they approach each other. This typically influences the achievable debit of data transfer in common communication solutions, which is a key aspect in data muling.

The described process can only be achieved if the mule AUV is able to locate the other vehicle and draw near it. For such application, USBL (Ultra-Short Baseline) systems prove to have several advantages comparatively to other localization methods, such as optical, radio and inertial based techniques. The main advantages are the achievable range, limited error and lower sensibility to environment conditions, such as salinity or turbidity. For that reason, in this scenario, a USBL system is used to receive the transmitted signals and calculate the angle of arrival of the acoustic signal, thus the direction that the mule AUV should navigate. Additionally, using a synchronization mechanism, the mule is also able to determine the distance to the acoustic source and thus the vehicles' relative positions.

In such scenario, the USBL system needs to meet specific requirements to assure a reliable localization. Since the acoustic source can be located anywhere, it is essential that the estimation is accurate for both short and long range distances. Additionally, the system needs to have line of sight in any direction, which is compromised from the start by deploying the sensors on an AUV. Typically, the available USBL commercial solutions do not tackle these issues simultaneously, so the development of such system constitutes a technological challenge.

Therefore, this dissertation intends to develop a method that improves relative localization of AUVs using reconfigurable USBL systems. All the contemplated tools and complementary mechanisms are carefully explained throughout the document. 

This research work falls under the scope of activities developed by the Center of Robotics and Autonomous Systems of INESC TEC. It is integrated in the GROW project which focuses on exploring the use of AUVs as data mules for long duration missions.

%-------------------------------------------------------------------------------------------
\section{Objectives} \label{sec:objective}

The goal of the present work is to study and propose an adaptive configuration selection method, which assumes the integration of several hydrophones in a USBL system to allow selecting the set of sensors that minimizes the estimation error. This aims to achieve high estimate accuracy for both short and long range distances and continuously provide a set of hydrophones that have line of sight with the target, which can be located anywhere. In order to attain this, a comparative study is developed on tools that allow to compare the performance of sensors configurations in order to select the most reliable option. Then, the proposed system is presented in detail and validated with comprehensive simulations.

Building upon previous developments on the USBL system, it is also intended to achieve a more rigorous calculation of the TDoA to enable a more precise localization. By associating this improved calculation with the correlation measurement already implemented, it is expected to obtain a more precise ToA measurement. 

%-------------------------------------------------------------------------------------------
\section{Document Structure}

The present document is partitioned into six chapters, which are summarized in this section.

Chapter \ref{chap:sota} offers an overview on background concepts about underwater acoustics, localization estimation and positioning systems, followed by USBL available commercial solutions and developed technology for a similar purpose. Then it focuses on angle of arrival determination methods and optimization mechanisms that are typically employed.

After reviewing the literature, chapter \ref{chap:problem} intends to clarify the problem that is being resolved in this thesis. The research hypothesis is stated as well as the research questions that are discussed and intended to be further explored. The chapter ends with the clarification of the used validation methods for the work. 

Chapter \ref{chap:proposed_sys} presents and explains the developed hardware design for the phase difference calculation. Then, three different approaches are presented for systematic comparison between the performance of a sensor configuration. These are supported with simulation experiments which allow to draw conclusions on the preferred approach.

Chapter \ref{chap:study} details the developed dynamically reconfigurable configuration method. The theoretical specifics and thought process are laid out and the mechanism is then validated through simulations.

Lastly, chapter \ref{chap:conclusion} gives the final remarks about the developed work and mentions research work which could be further developed in the future.  
