\chapter{Conclusions}  \label{chap:conclusion}

\section{Summary}

As stated at the beginning, the overall aim of this research work was to develop a USBL system with characteristics that lead to an improvement on localization precision, both for short and long range.

Firstly, it is presented the implementation of an hardware module that computes with precision the time differences of arrival of acoustic signals received by a set of four hydrophones. The architecture demonstrated a significant decrease of the used resources comparatively to the previous implementation and a precise calculation of the time differences of arrival. After initial field tests, a rough estimation of the obtained precision leads to conclude that the ToA measurement was improved by using the phase differences along with the correlation.

Secondly, an adaptive configuration selection method was developed which is capable of determining the optimal configuration from a discrete set of fixed hydrophones, leading to higher localization precision, i.e. lower estimation error. It uses a Monte Carlo approach for a systematic comparison between a high number of configurations and to obtain averaged estimations for more coherent results. Additionally, as an effort to improve the USBL system's viewing angle, it was integrated a mechanism that assures the continuous selection of hydrophone sets that have line of sight to the transmitter positions. The simulated results show that the method as proven to be successful, since the selected configurations always correspond to the layout that leads to the lower estimation error.

Additionally, as initially intended, the method was adjusted so that it was possible to conduct a comprehensive study on the configurations that demonstrated to be optimal for short and long range. It was concluded that using the FIM together with the preferred optimality criterion for this application, originated the most coherent results and allowed to elect the configurations that lead to an average higher precision.

This evaluation tool can be applied to any method that is capable of defining performance metrics for the evaluation of a sensor configuration. In the present research work three options were investigated, consisting on an estimator based on the TDoA measurement, an estimator that assumes a plane wavefront and the Crámer-Rao lower bound that uses the Fisher Information Matrix. From these three options, it was concluded that the most reliable would be the third due to the independence on the chosen estimator and its wide use on literature.

\section{Contributions }

The main contributions of this dissertation are: 
\begin{itemize}
	\item A HDL system design with strict area constraints that receives signals from four different sensors and calculates the phase differences between them;
	\item A comparative study on three methods that evaluate the performance of sensor configurations based on several different optimality criteria;
	\item The adaptive configuration selection method which is capable of determining the optimal sensor configuration, from a discrete set of hydrophones in fixed positions, that leads to highest localization precision, i.e. lowest estimation error.
\end{itemize}

%Additionally, a scientific article will be written and submitted to the OCEANS Conference that is occurring in May of 2021. 

\section{Future Work}

The developed work arose various innovative ideas and topics according to the state of the art on localization optimization methods. Therefore, there is the intent to produce a a scientific article on the subject to be submitted in the OCEANS Conference 2021.

The adopted optimization techniques in the present work are the starting point of a topic which is not very explored in the current literature. Consequently, there are several areas that can be improved as follows:

\begin{itemize}
	\item Implementing the developed algorithm for AoA estimation as a real-time system which can be used for decision making during underwater navigation;
	
	\item Using Machine Learning techniques in order to find the optimal hydrophone array configuration, to be positioned in an underwater vehicle;
	
	\item Integrating the estimator with a Kalman Filter or a Particle Swarm Optimization approach;
	
	\item Perform field experiments in order to examine if it is advantageous to select only the hydrophones with line of sight or if the errors obtained from propagation in the surface of the AUV are negligible;
\end{itemize}